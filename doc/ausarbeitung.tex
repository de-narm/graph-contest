%TODO FIND SOURCE FOR NP HARDNESS

\documentclass[]{article}

\usepackage{tikz}
\usepackage{verbatim}
\usepackage{biblatex}
\usepackage{filecontents}
\usepackage[absolute]{textpos}

\usetikzlibrary{positioning}

\begin{filecontents}{bib.bib}
@electronic{bollobas98,
  abstract = {The time has now come when graph theory should be part of the education of every serious student of mathematics and computer science, both for its own sake and to enhance the appreciation of mathematics as a whole. This book is an in-depth account of graph theory, written with such a student in mind; it reflects the current state of the subject and emphasizes connections with other branches of pure mathematics. The volume grew out of the author's earlier book, Graph Theory -- An Introductory Course, but its length is well over twice that of its predecessor, allowing it to reveal many exciting new developments in the subject. Recognizing that graph theory is one of several courses competing for the attention of a student, the book contains extensive descriptive passages designed to convey the flavor of the subject and to arouse interest. In addition to a modern treatment of the classical areas of graph theory such as coloring, matching, extremal theory, and algebraic graph theory, the book presents a detailed account of newer topics, including Szemer'edi's Regularity Lemma and its use, Shelah's extension of the Hales-Jewett Theorem, the precise nature of the phase transition in a random graph process, the connection between electrical networks and random walks on graphs, and the Tutte polynomial and its cousins in knot theory. In no other branch of mathematics is it as vital to tackle and solve challenging exercises in order to master the subject. To this end, the book contains an unusually large number of well thought-out exercises: over 600 in total. Although some are straightforward, most of them are substantial, and others will stretch even the most able reader.},
  added-at = {2015-05-19T09:19:57.000+0200},
  address = {New York},
  author = {Bollob\'{a}s, B\'{e}lla},
  biburl = {https://www.bibsonomy.org/bibtex/27b84cb41ef679da77300486d9f6f0916/ytyoun},
  doi = {10.1007/978-1-4612-0619-4},
  interhash = {875127e4e9980d87bf0bdeab7fa97559},
  intrahash = {7b84cb41ef679da77300486d9f6f0916},
  isbn = {9781461206194},
  keywords = {bollobas characteristic eigenvalues graph.theory hoffman monotonicity polynomial rayleigh textbook},
  publisher = {Springer},
  refid = {682118471},
  timestamp = {2017-11-10T06:33:22.000+0100},
  title = {Modern Graph Theory},
  year = 1998
}

@article{journals/corr/abs-1808-10519,
  added-at = {2018-09-04T00:00:00.000+0200},
  author = {Bekos, Michael A. and Förster, Henry and Geckeler, Christian and Holländer, Lukas and Kaufmann, Michael and Spallek, Amadäus M. and Splett, Jan},
  biburl = {https://www.bibsonomy.org/bibtex/2e13a59afe7afa2da70a3d34ebc7993c1/dblp},
  ee = {http://arxiv.org/abs/1808.10519},
  interhash = {0703b4c6a706aafb35707a22e420381f},
  intrahash = {e13a59afe7afa2da70a3d34ebc7993c1},
  journal = {CoRR},
  keywords = {dblp},
  timestamp = {2018-09-05T11:36:29.000+0200},
  title = {A Heuristic Approach towards Drawings of Graphs with High Crossing Resolution.},
  url = {http://dblp.uni-trier.de/db/journals/corr/corr1808.html#abs-1808-10519},
  volume = {abs/1808.10519},
  year = 2018
}

\end{filecontents}

\addbibresource{bib.bib}

\begin{document}
    \title{Graph Drawing Contest 2020 \\
           Crossing Minimization with Randomness}
    \author{Sebastian Benner}
    \maketitle
    
    \abstract{Stuff}

    \section{Introduction} 

	The annual Graph Drawing Contest\footnote{http://mozart.diei.unipg.it/gdcontest/contest2020/challenge.html} is an open challenge to design an algorithm for optimized graph drawing. The exact criteria for such a drawing are changed every couple of years, the current ones remain the same as last years challenge. The Live Challenge will contain between five to ten acyclic directed graphs with up to a few thousands nodes each. All resulting layouts must be submitted within one hour of the graphs being handed out.

The main criteria this time around are crossing which ought to be minimal in the resulting drawing. In itself this already poses a NP-hard problem. Additional constraints placed on the drawing are:
\begin{itemize}
	\item Each edge must be a straight upward facing line, meaning the source of each directed edge must be lower than the target.
	\item Each node must be placed upon a grid of given size.
	\item Crossings between a node and an edge are not permitted, as well as overlapping nodes.
\end{itemize}

After all graphs are collected, for each of the original graphs a best drawing is determined with all the other graphs receiving a weighted score based on the difference in crossings. The highest overall score wins the contest. Each team has to bring its own hardware to run their respective algorithm, meaning there is no limitation in terms of tools used and the given time to solve the task can be counterbalanced by more powerful hardware.

During the last couple of years most if not all of the top scoring contestants based their algorithm at least to some part on randomness which will be the basis of this work. The goal of is to evaluate different base drawings and to find a balance between lightweight calculations for random steps and a more directed approach to randomness.

	\section{Foundation}
	A $graph\;G$ is defined as an ordered pair $(V, E)$ of $vertices\;V$ and $edges\;E$. Edges are unordered pairs of two vertices $\{x,y\}$, said to $join$  them, and therefore $E$ is a subset of $V^{(2)}$. In the special case where edges are ordered pairs $(x,y)$ with $x$ as $source$ and $y$ as $target$ the graph $D$ is called $directed$ graph or $digraph$. We call the number of vertices the $order$ of $G$ and the number of edges the $size$ of $G$. $V(G)$ and $E(G)$ are the sets of vertices and edges of $G$ respectively, $x \in V(G)$ with vertex $x$ can be written as $x \in G$ while $\{x, y\} \in E(G)$ with unordered edge $\{x,y\}$ is written as $\{x,y\} \in G$. 

For a more comprehensive explanation I refer to the book $Modern$ $Graph$ $Theory$~\cite{bollobas98} upon which this notation is based on.

	\section{Prior Work}
	\paragraph{A Heuristic Approach towards Drawings of
Graphs with High Crossing Resolution~\cite{journals/corr/abs-1808-10519}}
	While not aiming at the same goal, this algorithm served as basis for the last years winner. The aim was to create a drawing with largest minimum angle, called the resolution of $G$,  possible. To achieve this the algorithm build and maintained two sets: all nodes and only such nodes that are deemed critical for the current resolution of $G$. Critical are all nodes connected with edges involved in minimal angles. In each iteration either a node from the set of all critical nodes is chosen uniformly or inverse proportionally to its proximity to a critical nodes from the set of all nodes. To determine the new position of a node rays are used which are cast out uniformly distributed in all directions with the best result becoming the new position. Combined with a energy-based base drawing and some tweaks to avoid local minima, the algorithm proved itself also won its respective year.

    \section{Sugiyama Framework}

    \section{Crossing Minimization}

    \section{Results}

    \printbibliography
\end{document}
